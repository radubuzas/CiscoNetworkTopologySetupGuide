\documentclass[a4paper,12pt]{article}
\usepackage{geometry}
\usepackage{setspace}
\usepackage{minted} % Requires pygments installed
\usepackage{titlesec}
\usepackage{xcolor}
\usepackage{graphicx}

\geometry{margin=1in}
\setstretch{1.1}
\definecolor{lightgray}{gray}{0.95}
\titleformat{\section}{\large\bfseries}{}{0em}{}

\begin{document}

\title{Cisco Packet Tracer Detailed Setup Guide}
\author{}
\date{}
\maketitle

\section*{HOST}
\begin{enumerate}
    \item Rename
    \item Turn off, add \texttt{*-CGE} Network Interface Card, turn on
    \item Config IP, mask, Default Gateway, DNS Server
    \item Config wireless profile
    \item Config e-mail address
    \item ????????? 
\end{enumerate}

\section*{SWITCH - 2960}
Rename the switch with the appropriate name. Usual format Sw\textbf{NetworkName}.

\vspace{0.3cm}

\noindent Basic configuration:

\begin{minted}[bgcolor=lightgray,fontsize=\small,breaklines]{bash}
> enable
# configure terminal
(config)# hostname SwNetworkName
(config)# no ip domain-lookup
(config)# no cdp run
(config)# service password-encryption
(config)# enable secret ciscosecpa55
(config)# enable password ciscoenapa55
(config)# banner motd "Informare generica"
\end{minted}

\noindent Setting up SSH:

\begin{minted}[bgcolor=lightgray,fontsize=\small,breaklines]{bash}
(config)# ip domain name sal.ro
(config)# username Admin01 privilege 15 secret Admin01pa55
(config)# crypto key generate rsa
2048
(config)# ip ssh version 2
(config)# ip ssh time-out 90
(config)# ip ssh authentication-retries 3
\end{minted}

\noindent Logging messages:

\begin{minted}[bgcolor=lightgray,fontsize=\small,breaklines]{bash}
(config)# logging host IpDnsServer
(config)# service timestamps log datetime msec
(config)# service timestamps debug datetime msec
\end{minted}

\noindent Console and VTY lines configuration
\begin{minted}[bgcolor=lightgray,fontsize=\small,breaklines]{bash}
(config)# line console 0
(config-line)# password ciscoconpa55
(config-line)# login
(config-line)# logging synchronous
(config-line)# exec-timeout 15 10
(config-line)# exit
(config)# line vty 0 15
(config-line)# password ciscovtypa55
(config-line)# login
(config-line)# logging synchronous
(config-line)# exec-timeout 10 15
(config-line)# transport input ssh
(config-line)# login local
(config-line)# exit
\end{minted}

\noindent Setting up the management interface:
\begin{minted}[bgcolor=lightgray,fontsize=\small,breaklines]{bash}
(config)# interface vlan 1
(config-if)# description "Relevant Switch Description"
(config-if)# ip address SwIpAddress NetworkMask
(config-if)# no shutdown
(config-if)# exit
(config)# ip default-gateway IpDefaultGateway
(config)# end
# clock set 12:34:56 01 Jan 2025
# copy running-config startup-config
\end{minted}

\section*{ROUTER - 2910}
Turn off the router and add \textbf{HWIC-2T} module for additional interfaces, as shown in Figure~\ref{fig:router}. Turn on the router.

\begin{figure}[h]
\centering
\includegraphics[width=0.8\textwidth]{./images/Router.png}
\caption{Router Image}
\label{fig:router}
\end{figure}

\vspace{0.3cm}

\noindent Rename the router with the appropriate name. Usual format R\textbf{NetworkName}.

\vspace{0.3cm}

\noindent Basic configuration:
\begin{minted}[bgcolor=lightgray,fontsize=\small,breaklines]{bash}
> enable
# configure terminal
(config)# hostname RNetworkName
(config)# no ip domain-lookup
(config)# no cdp run
(config)# service password-encryption
(config)# security passwords min-length 10
(config)# login block-for 60 attempts 3 within 15
(config)# enable secret ciscosecpa55
(config)# enable password ciscoenapa55
(config)# banner motd "Informare generica"
(config)# banner login "Consecinte legale de la stanga la dreapta pt persoanele neautorizate"
\end{minted}

\noindent Setting up SSH:
\begin{minted}[bgcolor=lightgray,fontsize=\small,breaklines]{bash}
(config)# ip domain name sal.ro
(config)# username Admin01 privilege 15 secret Admin01pa55
(config)# crypto key generate rsa
2048
(config)# ip ssh version 2
(config)# ip ssh time-out 90
(config)# ip ssh authentication-retries 3
\end{minted}

\noindent Logging messages:
\begin{minted}[bgcolor=lightgray,fontsize=\small,breaklines]{bash}
(config)# logging host IpDnsServer
(config)# service timestamps log datetime msec
(config)# service timestamps debug datetime msec
\end{minted}

\noindent Console and VTY lines configuration:
\begin{minted}[bgcolor=lightgray,fontsize=\small,breaklines]{bash}
(config)# line console 0 
(config-line)# password ciscoconpa55
(config-line)# login
(config-line)# logging synchronous
(config-line)# exec-timeout 15 10
(config-line)# exit
(config)# line vty 0 15
(config-line)# password ciscovtypa55
(config-line)# login
(config-line)# logging synchronous
(config-line)# exec-timeout 10 15
(config-line)# transport input ssh
(config-line)# login local
(config-line)# exit
\end{minted}

\noindent Setting up interfaces and routing:
\begin{minted}[bgcolor=lightgray,fontsize=\small,breaklines]{bash}
(config)# interface GigabitEthernet 0/0
(config-if)# description "Relevant Interface Description"
(config-if)# ip address RouterIpAddressLan NetworkMask
(config-if)# no shutdown
(config-if)# exit
(config)# interface Serial 0/0/0
(config-if)# description "Relevant Interface Description"
(config-if)# ip address RouterIpAddressWan NetworkMask
(config-if)# no shutdown
(config-if)# exit
\end{minted}

\noindent Routing Open Shortest Path First (\textbf{OSPF}) and disable interfaces that are not used by OSPF:
\begin{minted}[bgcolor=lightgray,fontsize=\small,breaklines]{bash}
(config)# router ospf 1
(config-router)# network WanIpAddress WildcardMask area 0
(config-router)# network LanIpAddress WildcardMask area 0
(config-router)# passive-interface GigabitEthernet 0/1
(config-router)# passive-interface GigabitEthernet 0/2
(config-router)# passive-interface Serial 0/0/1
(config-router)# end
\end{minted}

\noindent \textbf{Important note:} Used/unused interfaces may depend on the context, and this is just an example.

\vspace{0.3cm}

\noindent Set time and save configuration:
\begin{minted}[bgcolor=lightgray,fontsize=\small,breaklines]{bash}
# clock set 12:34:56 01 Jan 2025
# copy running-config startup-config 
\end{minted}

\section*{NTP setup}
\begin{minted}[bgcolor=lightgray,fontsize=\small,breaklines]{bash}
(config)# ntp server SERVER_IP
(config)# ntp authenticate
(config)# ntp trusted 1
(config)# ntp authentication-key md5 NTPa55
(config)# ntp update-calendar   (Only for Router)
(config)# show clock            (Check time)
\end{minted}

\end{document}
