\documentclass[a4paper,12pt]{article}
\usepackage{geometry}
\usepackage{setspace}
\usepackage{minted} % Requires pygments installed
\usepackage{titlesec}
\usepackage{xcolor}
\usepackage{graphicx}
\usepackage[hidelinks,bookmarksopen=true,bookmarksnumbered=true]{hyperref}

\geometry{margin=1in}
\setstretch{1.1}
\definecolor{lightgray}{gray}{0.95}
\titleformat{\section}{\large\bfseries}{\thesection}{0.8em}{}

\begin{document}

\title{Cisco Packet Tracer Detailed Setup Guide}
\author{}
\date{}
\maketitle

\tableofcontents

\section{Setup}
\label{sec:setup}

\subsection{Host / Server}
\begin{enumerate}
    \item Rename
    \item Turn off, add \texttt{*-CGE} Network Interface Card, turn on
    \item Configure IP, Mask, Default Gateway, DNS Server
    \item Config e-mail address
\end{enumerate}

\subsection{SWITCH - 2960}
Rename the switch with the appropriate name. Usual format Sw\textbf{NetworkName}.

\vspace{0.3cm}

\noindent Basic configuration:

\begin{minted}[bgcolor=lightgray,fontsize=\small,breaklines]{bash}
> enable
# configure terminal
(config)# hostname SwNetworkName
(config)# no ip domain-lookup
(config)# no cdp run
(config)# service password-encryption
(config)# enable secret ciscosecpa55
(config)# enable password ciscoenapa55
(config)# banner motd "Informare generica"
\end{minted}

\noindent Setting up SSH:

\begin{minted}[bgcolor=lightgray,fontsize=\small,breaklines]{bash}
(config)# ip domain name sal.ro
(config)# username Admin01 privilege 15 secret Admin01pa55
(config)# crypto key generate rsa
2048
(config)# ip ssh version 2
(config)# ip ssh time-out 90
(config)# ip ssh authentication-retries 3
\end{minted}

\noindent Logging messages:

\begin{minted}[bgcolor=lightgray,fontsize=\small,breaklines]{bash}
(config)# logging host IpDnsServer (e.g., 192.168.20.254)
(config)# service timestamps log datetime msec
(config)# service timestamps debug datetime msec
\end{minted}

\noindent Console and VTY lines configuration:
\begin{minted}[bgcolor=lightgray,fontsize=\small,breaklines]{bash}
(config)# line console 0
(config-line)# password ciscoconpa55
(config-line)# login
(config-line)# logging synchronous
(config-line)# exec-timeout 15 10
(config-line)# exit
(config)# line vty 0 15
(config-line)# password ciscovtypa55
(config-line)# login
(config-line)# logging synchronous
(config-line)# exec-timeout 10 15
(config-line)# transport input ssh
(config-line)# login local
(config-line)# exit
\end{minted}

\noindent Setting up the management interface:

\noindent \textbf{Note:} You may need to open some of the FastEthenert ports.
\begin{minted}[bgcolor=lightgray,fontsize=\small,breaklines]{bash}
(config)# interface vlan 1
(config-if)# description "Relevant Switch Description"
(config-if)# ip address SwIpAddress NetworkMask
(config-if)# no shutdown
(config-if)# exit
(config)# interface range fa0/1-24
(config-if-range)# shutdown
(config-if-range)# exit
(config)# ip default-gateway IpDefaultGateway
(config)# end
# clock set 12:34:56 01 Jan 2025
# copy running-config startup-config
\end{minted}

\subsection{ROUTER - 2910}
Turn off the router and add \textbf{HWIC-2T} module for additional interfaces, as shown in Figure~\ref{fig:router}. Turn on the router.

\begin{figure}[h]
\centering
\includegraphics[width=0.8\textwidth]{./images/Router.png}
\caption{Router Image}
\label{fig:router}
\end{figure}

\vspace{0.3cm}

\noindent Rename the router with the appropriate name. Usual format R\textbf{NetworkName}.

\vspace{0.3cm}

\noindent Basic configuration:
\begin{minted}[bgcolor=lightgray,fontsize=\small,breaklines]{bash}
> enable
# configure terminal
(config)# hostname RNetworkName
(config)# no ip domain-lookup
(config)# no cdp run
(config)# service password-encryption
(config)# security passwords min-length 10
(config)# login block-for 60 attempts 3 within 15
(config)# enable secret ciscosecpa55
(config)# enable password ciscoenapa55
(config)# banner motd "Informare generica"
(config)# banner login "Consecinte legale de la stanga la dreapta pt persoanele neautorizate"
\end{minted}

\noindent Setting up SSH:
\begin{minted}[bgcolor=lightgray,fontsize=\small,breaklines]{bash}
(config)# ip domain name sal.ro
(config)# username Admin01 privilege 15 secret Admin01pa55
(config)# crypto key generate rsa
2048
(config)# ip ssh version 2
(config)# ip ssh time-out 90
(config)# ip ssh authentication-retries 3
\end{minted}

\noindent Logging messages:
\begin{minted}[bgcolor=lightgray,fontsize=\small,breaklines]{bash}
(config)# logging host IpDnsServer (e.g., 192.168.20.254)
(config)# service timestamps log datetime msec
(config)# service timestamps debug datetime msec
\end{minted}

\noindent Console and VTY lines configuration:
\begin{minted}[bgcolor=lightgray,fontsize=\small,breaklines]{bash}
(config)# line console 0 
(config-line)# password ciscoconpa55
(config-line)# login
(config-line)# logging synchronous
(config-line)# exec-timeout 15 10
(config-line)# exit
(config)# line vty 0 15
(config-line)# password ciscovtypa55
(config-line)# login
(config-line)# logging synchronous
(config-line)# exec-timeout 10 15
(config-line)# transport input ssh
(config-line)# login local
(config-line)# exit
\end{minted}

\noindent Setting up interfaces and routing:
\begin{minted}[bgcolor=lightgray,fontsize=\small,breaklines]{bash}
(config)# interface GigabitEthernet 0/0
(config-if)# description "Relevant Interface Description"
(config-if)# ip address RouterIpAddressLan NetworkMask
(config-if)# no shutdown
(config-if)# exit
(config)# interface Serial 0/0/0
(config-if)# description "Relevant Interface Description"
(config-if)# ip address RouterIpAddressWan NetworkMask
(config-if)# no shutdown
(config-if)# exit
\end{minted}

\noindent Routing Open Shortest Path First (\textbf{OSPF}) and disable interfaces that are not used by OSPF:
\begin{minted}[bgcolor=lightgray,fontsize=\small,breaklines]{bash}
(config)# router ospf 1
(config-router)# network WanIpAddress WildcardMask area 0
(config-router)# network LanIpAddress WildcardMask area 0
(config-router)# passive-interface GigabitEthernet 0/1
(config-router)# passive-interface GigabitEthernet 0/2
(config-router)# passive-interface Serial 0/0/1
(config-router)# end
\end{minted}

\noindent \textbf{Important note:} Used/unused interfaces may depend on the context, and this is just an example.

\vspace{0.3cm}

\noindent Set time and save configuration:
\begin{minted}[bgcolor=lightgray,fontsize=\small,breaklines]{bash}
# clock set 12:34:56 01 Jan 2025
# copy running-config startup-config 
\end{minted}

\section{Services}

\subsection{DHCP}

\noindent \textbf{RouterName (DHCP Server - Donald):}

\noindent \textbf{Note:} Set IpExcludedStart and IpExcludedEnd to exclude the address range from the default gateway to the host IP (i.e., IpExcludedStart = IpDefaultGateway, IpExcludedEnd = HostIP).
\begin{minted}[bgcolor=lightgray,fontsize=\small,breaklines]{bash}
(config)# ip dhcp excluded-address IpExcludedStart IpExcludedEnd
(config)# ip dhcp pool PoolName
(dhcp-config)# network NetworkAddress NetworkMask
(dhcp-config)# default-router IpDefaultGateway
(dhcp-config)# dns-server IpDnsServer (e.g., 192.168.20.254)
(dhcp-config)# domain-name DomainName
\end{minted}

\noindent \textbf{RouterName (DHCP Relay - Daisy / Duck):}

\noindent \textbf{Note:} \textit{IpDefaultGateway} should be of type 192.168.*.* and \textit{IpDhcpServer} should be of type 10.10.10.*
\begin{minted}[bgcolor=lightgray,fontsize=\small,breaklines]{bash}
(config)# interface InterfaceName
(config-if)# ip helper-address IpDhcpServer
\end{minted}

\noindent After configuration the DHCP server, check the result by bringing another PC (e.g., Daisy-DHCP) to the topology and assign its IP via DHCP. See Figure~\ref{fig:dhcp}.

\begin{figure}[h!]
\centering
\includegraphics[width=0.8\textwidth]{./images/Dhcp.png}
\caption{Daisy DHCP IP assignation}
\label{fig:dhcp}
\end{figure}

\subsection{NTP}

\noindent \textbf{Switches and Routers:}
\begin{minted}[bgcolor=lightgray,fontsize=\small,breaklines]{bash}
(config)# ntp server IpNtpServer
(config)# ntp authenticate
(config)# ntp trusted-key 1
(config)# ntp authentication-key 1 MD5 NTPpa55
\end{minted}

\noindent \textbf{Routers only:}
\begin{minted}[bgcolor=lightgray,fontsize=\small,breaklines]{bash}
(config)# ntp update-calendar
\end{minted}

\noindent \textbf{Server:}

\noindent Make sure to turn it \textbf{ON}.

\begin{itemize}
    \item \textbf{Authentication:} \texttt{Enable}
    \item \textbf{Key:} \texttt{1}
    \item \textbf{Password:} NTPpa55
\end{itemize}

\subsection{FTP}

\noindent \textbf{Command Prompt:}
\begin{minted}[bgcolor=lightgray,fontsize=\small,breaklines]{bash}
ftp IpFtpServer
dir
get FileName
\end{minted}

\subsection{HTTP / HTTPS}

\noindent In the server configuration window:

\begin{itemize}
    \item Set \textbf{HTTP = OFF}
    \item Set \textbf{HTTPS = ON}
\end{itemize}

\subsection{DNS}

\noindent Configure DNS on the server:

\noindent Make sure to turn it \textbf{ON}.

\begin{itemize}
    \item \textbf{Name:} \texttt{sal.ro}
    \item \textbf{Type:} \texttt{A Record}
    \item \textbf{Address:} IpDnsServer (e.g., \texttt{192.168.20.254})
\end{itemize}

\noindent After adding the DNS entry, verify from a PC using a web browser:

\noindent Search from its browser \texttt{https://sal.ro}

\subsection{SYSLOG}

\noindent Enable SYSLOG on the server. See Figure~\ref{fig:syslog}.

\begin{figure}[h!]
\centering
\includegraphics[width=0.8\textwidth]{./images/Syslog.png}
\caption{Syslog Image}
\label{fig:syslog}
\end{figure}

\noindent Configure the router/switch to send logs. This step should be covered in section \hyperref[sec:setup]{Setup}.

\subsection{EMAIL}

\noindent Configure the email account in the PC (Desktop $\rightarrow$ Email). See Figure~\ref{fig:emailhost}. 

\noindent \textbf{Password}: \texttt{123456}

\begin{figure}[h!]
\centering
\includegraphics[width=0.8\textwidth]{./images/EmailHost.png}
\caption{Email Host Image}
\label{fig:emailhost}
\end{figure}


Then configure the Email server settings:

\begin{itemize}
    \item Username and password as required. Password \texttt{123456}
    \item POP3 \& SMTP enabled
\end{itemize}

\subsection{AAA}

\noindent \textbf{On the Server (Services $\rightarrow$ AAA):}

\begin{itemize}
    \item Set \textbf{Service: On}
    \item Verify \textbf{Radius Port}: \texttt{1645}
\end{itemize}

\noindent \textbf{Network Configuration:}

\begin{itemize}
    \item \textbf{Client Name:} Wi-FiDaisy
    \item \textbf{Client IP:} 10.10.10.2
    \item \textbf{Server Type:} Radius
    \item \textbf{Secret:} \texttt{RadiusPa55}
    \item Press \textbf{Add}
\end{itemize}

\noindent \textbf{User Setup:}

\begin{itemize}
    \item \textbf{Username:} Wi-FiDaisy
    \item \textbf{Password:} \texttt{RadiusPa55}
    \item Press \textbf{Add}
\end{itemize}

\noindent AAA is now active and ready to authenticate RADIUS clients.

% ==========================================
% SECTION 3: WI-FI
% ==========================================
\section{Wi-Fi}

\subsection{Physical Connection}
\begin{enumerate}
    \item Add a \textbf{WRT300N} Wireless Router.
    \item Connect a configuration laptop (Service Laptop) to the router using a \textbf{Copper Straight-Through} cable (Laptop FastEthernet0 $\rightarrow$ Router Ethernet 1).
    \item Connect the Router to the ISP/Main Router using the \textbf{Internet} port.
\end{enumerate}

\subsection{Accessing the GUI}
\begin{enumerate}
    \item On the Service Laptop, configure a static IP (e.g., \texttt{192.168.0.10}) in the same range as the default router IP (\texttt{192.168.0.1}).
    \item Open the Web Browser and navigate to \texttt{192.168.0.1}.
    \item Login with username: \texttt{admin} and password: \texttt{admin}.
\end{enumerate}

\subsection{Network Setup (GUI)}
\begin{itemize}
    \item \textbf{Internet Connection Type:} Static IP
    \item \textbf{Internet IP Address:} \texttt{10.10.10.2} (Interface facing the ISP)
    \item \textbf{Subnet Mask:} \texttt{255.255.255.252}
    \item \textbf{Default Gateway:} \texttt{10.10.10.1}
    \item \textbf{DNS 1:} IpDnsServer (e.g., \texttt{192.168.20.254})
    \item \textbf{Router IP:} Change Local IP if required (e.g., \texttt{192.168.50.65}).
    \item \textbf{DHCP Server:} Enabled
    \item Click \textbf{Save Settings}.
\end{itemize}

\noindent \textbf{Note:} If you changed the Router IP, you must release/renew the IP on your Service Laptop to reconnect to the GUI.

\subsection{Wireless Settings} 
\begin{itemize}
    \item Navigate to \textbf{Wireless} $\rightarrow$ \textbf{Basic Wireless Settings}.
    \item \textbf{Network Name (SSID):} \texttt{Wi-FiDaisy}
    \item \textbf{Standard Channel:} \texttt{6 or 11}
    \item \textbf{SSID Broadcast:} Enabled
    \item Click \textbf{Save Settings}.
\end{itemize}

\subsection{Wireless Security (WPA2 Enterprise)}
\begin{itemize}
    \item Navigate to \textbf{Wireless} $\rightarrow$ \textbf{Wireless Security}.
    \item \textbf{Security Mode:} WPA2 Enterprise
    \item \textbf{Encryption:} AES
    \item \textbf{RADIUS Server:} IpRadiusServer (e.g., \texttt{192.168.20.254})
    \item \textbf{Shared Secret:} \texttt{RadiusPa55}
    \item Click \textbf{Save Settings}.
\end{itemize}

\subsection{Wireless Client (Laptop) Setup}
\begin{enumerate}
    \item Turn off laptop, replace module with \textbf{WPC300N}, turn on.
    \item Open \textbf{PC Wireless} software on Desktop.
    \item Go to \textbf{Profiles} $\rightarrow$ \textbf{New} $\rightarrow$ Enter Profile Name (e.g., Wi-FiDaisy).
    \item \textbf{Advanced Setup} $\rightarrow$ select Infrastructure Mode.
    \item \textbf{Security:} WPA2 Enterprise.
    \item \textbf{Login Name:} \texttt{Wi-FiDaisy}
    \item \textbf{Password:} \texttt{RadiusPa55} (from User Setup in AAA).
    \item Click \textbf{Connect}.
\end{enumerate}

\subsection{MAC Address Filtering}
\begin{enumerate}
    \item Identify the MAC address of the laptop to block (command: \texttt{ipconfig /all}).
    \item In Router GUI, go to \textbf{Wireless} $\rightarrow$ \textbf{Wireless MAC Filter}.
    \item Select \textbf{Enabled} and \textbf{Prevent PCs listed below}.
    \item Enter the MAC address (format: \texttt{00:11:22:33:44:55}).
    \item Click \textbf{Save Settings}.
\end{enumerate}

\subsection{Configuring the ISP Router Interface} 
Connect to the router (e.g., RDaisy) connected to the Wi-Fi router.

\begin{minted}[bgcolor=lightgray,fontsize=\small,breaklines]{bash}
(config)# interface GigabitEthernet 0/1
(config-if)# description "Connection to Wi-Fi Router"
(config-if)# ip address 10.10.10.1 255.255.255.252
(config-if)# no shutdown
(config-if)# exit
\end{minted}

\noindent Add the Wi-Fi network to OSPF routing:

\begin{minted}[bgcolor=lightgray,fontsize=\small,breaklines]{bash}
(config)# router ospf 1
(config-router)# network 10.10.10.0 0.0.0.3 area 0
\end{minted}

\end{document}